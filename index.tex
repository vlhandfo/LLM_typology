% Options for packages loaded elsewhere
\PassOptionsToPackage{unicode}{hyperref}
\PassOptionsToPackage{hyphens}{url}
\PassOptionsToPackage{dvipsnames,svgnames,x11names}{xcolor}
%
\documentclass[
  letterpaper,
  DIV=11,
  numbers=noendperiod]{scrreprt}

\usepackage{amsmath,amssymb}
\usepackage{iftex}
\ifPDFTeX
  \usepackage[T1]{fontenc}
  \usepackage[utf8]{inputenc}
  \usepackage{textcomp} % provide euro and other symbols
\else % if luatex or xetex
  \usepackage{unicode-math}
  \defaultfontfeatures{Scale=MatchLowercase}
  \defaultfontfeatures[\rmfamily]{Ligatures=TeX,Scale=1}
\fi
\usepackage{lmodern}
\ifPDFTeX\else  
    % xetex/luatex font selection
\fi
% Use upquote if available, for straight quotes in verbatim environments
\IfFileExists{upquote.sty}{\usepackage{upquote}}{}
\IfFileExists{microtype.sty}{% use microtype if available
  \usepackage[]{microtype}
  \UseMicrotypeSet[protrusion]{basicmath} % disable protrusion for tt fonts
}{}
\makeatletter
\@ifundefined{KOMAClassName}{% if non-KOMA class
  \IfFileExists{parskip.sty}{%
    \usepackage{parskip}
  }{% else
    \setlength{\parindent}{0pt}
    \setlength{\parskip}{6pt plus 2pt minus 1pt}}
}{% if KOMA class
  \KOMAoptions{parskip=half}}
\makeatother
\usepackage{xcolor}
\setlength{\emergencystretch}{3em} % prevent overfull lines
\setcounter{secnumdepth}{5}
% Make \paragraph and \subparagraph free-standing
\makeatletter
\ifx\paragraph\undefined\else
  \let\oldparagraph\paragraph
  \renewcommand{\paragraph}{
    \@ifstar
      \xxxParagraphStar
      \xxxParagraphNoStar
  }
  \newcommand{\xxxParagraphStar}[1]{\oldparagraph*{#1}\mbox{}}
  \newcommand{\xxxParagraphNoStar}[1]{\oldparagraph{#1}\mbox{}}
\fi
\ifx\subparagraph\undefined\else
  \let\oldsubparagraph\subparagraph
  \renewcommand{\subparagraph}{
    \@ifstar
      \xxxSubParagraphStar
      \xxxSubParagraphNoStar
  }
  \newcommand{\xxxSubParagraphStar}[1]{\oldsubparagraph*{#1}\mbox{}}
  \newcommand{\xxxSubParagraphNoStar}[1]{\oldsubparagraph{#1}\mbox{}}
\fi
\makeatother


\providecommand{\tightlist}{%
  \setlength{\itemsep}{0pt}\setlength{\parskip}{0pt}}\usepackage{longtable,booktabs,array}
\usepackage{calc} % for calculating minipage widths
% Correct order of tables after \paragraph or \subparagraph
\usepackage{etoolbox}
\makeatletter
\patchcmd\longtable{\par}{\if@noskipsec\mbox{}\fi\par}{}{}
\makeatother
% Allow footnotes in longtable head/foot
\IfFileExists{footnotehyper.sty}{\usepackage{footnotehyper}}{\usepackage{footnote}}
\makesavenoteenv{longtable}
\usepackage{graphicx}
\makeatletter
\newsavebox\pandoc@box
\newcommand*\pandocbounded[1]{% scales image to fit in text height/width
  \sbox\pandoc@box{#1}%
  \Gscale@div\@tempa{\textheight}{\dimexpr\ht\pandoc@box+\dp\pandoc@box\relax}%
  \Gscale@div\@tempb{\linewidth}{\wd\pandoc@box}%
  \ifdim\@tempb\p@<\@tempa\p@\let\@tempa\@tempb\fi% select the smaller of both
  \ifdim\@tempa\p@<\p@\scalebox{\@tempa}{\usebox\pandoc@box}%
  \else\usebox{\pandoc@box}%
  \fi%
}
% Set default figure placement to htbp
\def\fps@figure{htbp}
\makeatother
% definitions for citeproc citations
\NewDocumentCommand\citeproctext{}{}
\NewDocumentCommand\citeproc{mm}{%
  \begingroup\def\citeproctext{#2}\cite{#1}\endgroup}
\makeatletter
 % allow citations to break across lines
 \let\@cite@ofmt\@firstofone
 % avoid brackets around text for \cite:
 \def\@biblabel#1{}
 \def\@cite#1#2{{#1\if@tempswa , #2\fi}}
\makeatother
\newlength{\cslhangindent}
\setlength{\cslhangindent}{1.5em}
\newlength{\csllabelwidth}
\setlength{\csllabelwidth}{3em}
\newenvironment{CSLReferences}[2] % #1 hanging-indent, #2 entry-spacing
 {\begin{list}{}{%
  \setlength{\itemindent}{0pt}
  \setlength{\leftmargin}{0pt}
  \setlength{\parsep}{0pt}
  % turn on hanging indent if param 1 is 1
  \ifodd #1
   \setlength{\leftmargin}{\cslhangindent}
   \setlength{\itemindent}{-1\cslhangindent}
  \fi
  % set entry spacing
  \setlength{\itemsep}{#2\baselineskip}}}
 {\end{list}}
\usepackage{calc}
\newcommand{\CSLBlock}[1]{\hfill\break\parbox[t]{\linewidth}{\strut\ignorespaces#1\strut}}
\newcommand{\CSLLeftMargin}[1]{\parbox[t]{\csllabelwidth}{\strut#1\strut}}
\newcommand{\CSLRightInline}[1]{\parbox[t]{\linewidth - \csllabelwidth}{\strut#1\strut}}
\newcommand{\CSLIndent}[1]{\hspace{\cslhangindent}#1}

\KOMAoption{captions}{tableheading}
\makeatletter
\@ifpackageloaded{tcolorbox}{}{\usepackage[skins,breakable]{tcolorbox}}
\@ifpackageloaded{fontawesome5}{}{\usepackage{fontawesome5}}
\definecolor{quarto-callout-color}{HTML}{909090}
\definecolor{quarto-callout-note-color}{HTML}{0758E5}
\definecolor{quarto-callout-important-color}{HTML}{CC1914}
\definecolor{quarto-callout-warning-color}{HTML}{EB9113}
\definecolor{quarto-callout-tip-color}{HTML}{00A047}
\definecolor{quarto-callout-caution-color}{HTML}{FC5300}
\definecolor{quarto-callout-color-frame}{HTML}{acacac}
\definecolor{quarto-callout-note-color-frame}{HTML}{4582ec}
\definecolor{quarto-callout-important-color-frame}{HTML}{d9534f}
\definecolor{quarto-callout-warning-color-frame}{HTML}{f0ad4e}
\definecolor{quarto-callout-tip-color-frame}{HTML}{02b875}
\definecolor{quarto-callout-caution-color-frame}{HTML}{fd7e14}
\makeatother
\makeatletter
\@ifpackageloaded{bookmark}{}{\usepackage{bookmark}}
\makeatother
\makeatletter
\@ifpackageloaded{caption}{}{\usepackage{caption}}
\AtBeginDocument{%
\ifdefined\contentsname
  \renewcommand*\contentsname{Table of contents}
\else
  \newcommand\contentsname{Table of contents}
\fi
\ifdefined\listfigurename
  \renewcommand*\listfigurename{List of Figures}
\else
  \newcommand\listfigurename{List of Figures}
\fi
\ifdefined\listtablename
  \renewcommand*\listtablename{List of Tables}
\else
  \newcommand\listtablename{List of Tables}
\fi
\ifdefined\figurename
  \renewcommand*\figurename{Figure}
\else
  \newcommand\figurename{Figure}
\fi
\ifdefined\tablename
  \renewcommand*\tablename{Table}
\else
  \newcommand\tablename{Table}
\fi
}
\@ifpackageloaded{float}{}{\usepackage{float}}
\floatstyle{ruled}
\@ifundefined{c@chapter}{\newfloat{codelisting}{h}{lop}}{\newfloat{codelisting}{h}{lop}[chapter]}
\floatname{codelisting}{Listing}
\newcommand*\listoflistings{\listof{codelisting}{List of Listings}}
\makeatother
\makeatletter
\makeatother
\makeatletter
\@ifpackageloaded{caption}{}{\usepackage{caption}}
\@ifpackageloaded{subcaption}{}{\usepackage{subcaption}}
\makeatother

\usepackage{bookmark}

\IfFileExists{xurl.sty}{\usepackage{xurl}}{} % add URL line breaks if available
\urlstyle{same} % disable monospaced font for URLs
\hypersetup{
  pdftitle={The Untitled Victoria Project},
  pdfauthor={Victoria Handford},
  colorlinks=true,
  linkcolor={blue},
  filecolor={Maroon},
  citecolor={Blue},
  urlcolor={Blue},
  pdfcreator={LaTeX via pandoc}}


\title{The Untitled Victoria Project}
\author{Victoria Handford}
\date{2025-03-14}

\begin{document}
\maketitle

\renewcommand*\contentsname{Table of contents}
{
\hypersetup{linkcolor=}
\setcounter{tocdepth}{2}
\tableofcontents
}

\bookmarksetup{startatroot}

\chapter*{Abstract}\label{abstract}
\addcontentsline{toc}{chapter}{Abstract}

\markboth{Abstract}{Abstract}

Submission for \href{https://conll.org/}{CoNLL 2025}.

\bookmarksetup{startatroot}

\chapter{Introduction}\label{introduction}

In recent years, Transformer-based large language models (LLMs) have
become a dominant paradigm in NLP, improving the performance of
applications in many language-related tasks. Apart from the general
assumption that model performance improves as the amount of training
data increases, the precise relationship between the size of the dataset
and task-specific performance gains is not well-defined.

This study investigates the learning process of monolingual LLMs trained
on a set of typologically diverse languages. Specifically, we evaluate
intermediate training checkpoints on a set of morphosyntactic tasks to
gain insight into the relationship between the amount of training data
and the models' performance on foundational linguistic tasks.

While languages like English display relatively uniform learning curves
across all tasks, others show marked task-specific variations,
highlighting the importance of taking linguistic aspects into account.
We observe variation in ranges of performance gains across languages and
identify typological features correlated with the amount of improvement.
Likewise, we see that models trained on different languages learn at
different rates, and there are typological features correlated to the
point at which the model reaches 90\% of its overall performance gains,
estimated with a best-fit curve.

\bookmarksetup{startatroot}

\chapter{Related Work}\label{related-work}

Zhang et al. (2021)

\section{Typological Databases}\label{typological-databases}

\subsection{WALS}\label{wals}

Dryer and Haspelmath (2013)\\
The World Atlas of Language Structures (WALS) is a collection of
linguistic features documenting the structural elements of many of the
world's languages. This database contains data on a variety of
properties including, but not limited to, phonology, grammar, lexicon,
language family, and the general location of where the language is most
spoken. Since languages are grouped into categories based on what they
predominately exhibit, it is a useful resource for exploring the
typological features of languages and cross-linguistic patterns (Dryer
and Haspelmath (2013)).

\subsection{Grambank}\label{grambank}

Skirgård et al. (2023)\\
Grambank is another database that compiles various grammatical features
for cross-linguistic analyses. Most of the features are structured as
questions where the corresponding feature values are coded as 0 if the
answer is no and 1 if yes. This database and its encodings are the
foundation for vector representations of the languages.

\bookmarksetup{startatroot}

\chapter{Models and Data}\label{models-and-data}

\bookmarksetup{startatroot}

\chapter{Methods}\label{methods}

\bookmarksetup{startatroot}

\chapter{Results}\label{results}

\section{Learning Curve Variation}\label{learning-curve-variation}

\subsection{Language-Specific}\label{language-specific}

\subsection{Task-Specific}\label{task-specific}

\section{Delta Variation}\label{delta-variation}

\section{Typological Correlations}\label{typological-correlations}

\bookmarksetup{startatroot}

\chapter{Discussion}\label{discussion}

\bookmarksetup{startatroot}

\chapter{Conclusion}\label{conclusion}

\begin{tcolorbox}[enhanced jigsaw, coltitle=black, colbacktitle=quarto-callout-important-color!10!white, left=2mm, titlerule=0mm, arc=.35mm, bottomrule=.15mm, bottomtitle=1mm, title=\textcolor{quarto-callout-important-color}{\faExclamation}\hspace{0.5em}{Work in Progress}, toptitle=1mm, breakable, opacityback=0, colback=white, opacitybacktitle=0.6, toprule=.15mm, rightrule=.15mm, colframe=quarto-callout-important-color-frame, leftrule=.75mm]

\end{tcolorbox}

\bookmarksetup{startatroot}

\chapter*{References}\label{references}
\addcontentsline{toc}{chapter}{References}

\markboth{References}{References}

\phantomsection\label{refs}
\begin{CSLReferences}{1}{0}
\bibitem[\citeproctext]{ref-wals}
Dryer, Matthew S., and Martin Haspelmath, eds. 2013. \emph{WALS Online
(V2020.3)}. Data set. Zenodo.
\url{https://doi.org/10.5281/zenodo.7385533}.

\bibitem[\citeproctext]{ref-grambank_dataset_zenodo_v1}
Skirgård, Hedvig, Hannah J. Haynie, Harald Hammarström, Damián E. Blasi,
Jeremy Collins, Jay Latarche, Jakob Lesage, et al. 2023. {``Grambank
V1.0.''} Zenodo. \url{https://doi.org/10.5281/zenodo.7740140}.

\bibitem[\citeproctext]{ref-Zhang_Warstadt_Li_Bowman_2021}
Zhang, Yian, Alex Warstadt, Xiaocheng Li, and Samuel R. Bowman. 2021.
{``When Do You Need Billions of Words of Pretraining Data?''} In
\emph{Proceedings of the 59th Annual Meeting of the Association for
Computational Linguistics and the 11th International Joint Conference on
Natural Language Processing (Volume 1: Long Papers)}, 1112--25. Online:
Association for Computational Linguistics.
\url{https://doi.org/10.18653/v1/2021.acl-long.90}.

\end{CSLReferences}

\bookmarksetup{startatroot}

\chapter*{Appendix}\label{appendix}
\addcontentsline{toc}{chapter}{Appendix}

\markboth{Appendix}{Appendix}

\section*{All Features}\label{all-features}
\addcontentsline{toc}{section}{All Features}

\markright{All Features}

\section*{Raw P90 Thresholds}\label{raw-p90-thresholds}
\addcontentsline{toc}{section}{Raw P90 Thresholds}

\markright{Raw P90 Thresholds}

\section*{Raw Delta Values}\label{raw-delta-values}
\addcontentsline{toc}{section}{Raw Delta Values}

\markright{Raw Delta Values}




\end{document}
